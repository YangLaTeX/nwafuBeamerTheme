%% 自定义相关的名称宏命令
%% ==================================================
%% \newcommand{\yourcommand}[参数个数]{内容}
% 西北农林科技大学各单位名称
\newcommand{\nwafu}{西北农林科技大学}
\newcommand{\cie}{信息工程学院}
\newcommand{\ca}{农学院}
\newcommand{\cpp}{植物保护学院}
\newcommand{\ch}{园艺学院}
\newcommand{\cast}{动物科技学院}
\newcommand{\cvm}{动物医学院}
\newcommand{\cf}{林学院}
\newcommand{\claa}{风景园林艺术学院}
\newcommand{\cnre}{资源环境学院}
\newcommand{\cwrae}{水利与建筑工程学院}
\newcommand{\cmee}{机械与电子工程学院}
\newcommand{\cfse}{食品科学与工程学院}
\newcommand{\ce}{葡萄酒学院}
\newcommand{\cls}{生命科学学院}
\newcommand{\cs}{理学院}
\newcommand{\ccp}{化学与药学院}
\newcommand{\cem}{经济管理学院}
\newcommand{\cm}{马克思主义学院}
\newcommand{\dfl}{外语系}
\newcommand{\iec}{创新实验学院}
\newcommand{\ci}{国际学院}
\newcommand{\dpe}{体育部}
\newcommand{\cvae}{成人教育}
\newcommand{\iswc}{水土保持研究所}

%% 签署春秋学期日期命令
\newcommand{\tomonth}{
  \the\year 年\the\month 月
}


\newcommand{\tomonthen}{
  \ifcase\the\month
  \or January%
  \or February%
  \or March%
  \or April%
  \or May%
  \or June%
  \or July%
  \or August%
  \or September%
  \or October%
  \or November%
  \or December%
  \fi, \the\year
}

\newcommand{\tosemester}{
  \the\year 年\ 
  \ifcase\the\month
  \or 秋%
  \or 春%
  \or 春%
  \or 春%
  \or 春%
  \or 春%
  \or 春%
  \or 夏%
  \or 秋%
  \or 秋%
  \or 秋%
  \or 秋%
  \fi 
}

\newcommand{\tosemesteren}{  
  \ifcase\the\month
  \or Autumn%
  \or Spring%
  \or Spring%
  \or Spring%
  \or Spring%
  \or Spring%
  \or Summer%
  \or Autumn%
  \or Autumn%
  \or Autumn%
  \or Autumn%
  \or Autumn%
  \fi, \the\year
}

% 插图路径设置
% ==================================================
\graphicspath{{figs/}}%图片所在的目录
% ==================================================

% 为标题页/封底页指定一个 logo
\pgfdeclareimage[height=0.5cm]{titlepagelogo}{nwafulogo/h_bar}% 标题页
\titlegraphic{% 标题页底部
  \pgfuseimage{titlepagelogo}
}

% 载入需要的TiKZ库
\usetikzlibrary{chains}

%% 设置绘制目录结构的宏及参数
\usepackage[edges]{forest}
\definecolor{folderbg}{RGB}{124,166,198}
\definecolor{folderborder}{RGB}{110,144,169}
\newlength\Size
\setlength\Size{4pt}
\tikzset{%
  folder/.pic={%
    \filldraw [draw=folderborder, top color=folderbg!50, bottom color=folderbg] (-1.05*\Size,0.2\Size+5pt) rectangle ++(.75*\Size,-0.2\Size-5pt);
    \filldraw [draw=folderborder, top color=folderbg!50, bottom color=folderbg] (-1.15*\Size,-\Size) rectangle (1.15*\Size,\Size);
  },
  file/.pic={%
    \filldraw [draw=folderborder, top color=folderbg!5, bottom color=folderbg!10] (-\Size,.4*\Size+5pt) coordinate (a) |- (\Size,-1.2*\Size) coordinate (b) -- ++(0,1.6*\Size) coordinate (c) -- ++(-5pt,5pt) coordinate (d) -- cycle (d) |- (c) ;
  },
}
\forestset{%
  declare autowrapped toks={pic me}{},
  declare boolean register={pic root},
  pic root=0,
  pic dir tree/.style={%
    for tree={%
      folder,
      %font=\ttfamily,
      grow'=0,
      s sep=1.0pt,
      font=\small \sffamily,
      %fit=band,
      %ysep = 1.0pt,
      inner ysep = 2.6pt,
    },
    before typesetting nodes={%
      for tree={%
        edge label+/.option={pic me},
      },
      if pic root={
        tikz+={
          \pic at ([xshift=\Size].west) {folder};
        },
        align={l}
      }{},
    },
  },
  pic me set/.code n args=2{%
    \forestset{%
      #1/.style={%
        inner xsep=2\Size,
        pic me={pic {#2}},
      }
    }
  },
  pic me set={directory}{folder},
  pic me set={file}{file},  
}
%% ==================================================

%%% Local Variables: 
%%% mode: latex
%%% TeX-master: "../main.tex"
%%% End: 
