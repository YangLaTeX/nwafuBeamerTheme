%\documentclass[fontset = none, t, aspectratio=169]{ctexbeamer}
\documentclass[fontset = none, t, aspectratio=169]{ctexbeamer}

% 修改自萧山主题的凤岗主题
\usetheme{nwafufengang}

% 载入需要的宏包
% 加载宏包
%===================注意======================%
% 在调用beamer.cls宏包后,以下宏包将自动调用,
% 不应单独调用这些宏包,以免发生冲突
% amsfonts, amsmath, amssymb, amsthm, 
% enumerate, geometry, graphics, graphicx, 
% hyperref, url, 
% ifpdf, keyval, xcolor, xxcolor
% =============================================%

% 参考文献宏包
% \usepackage[backend=biber,autolang=hyphen,style=gb7714-2015ay,
% doi=false,url=false,isbn=false]{biblatex}
\usepackage[backend=biber,autolang=hyphen,style=gb7714-2015,
gbtype=true,gbalign=gb7714-2015,
doi=false,url=false,isbn=false]{biblatex}

% Extra packages for the demo:
\usepackage{booktabs}
\usepackage{colortbl}
\usepackage{ragged2e}
\usepackage{schemabloc}

% 加载需要的宏包
\usepackage{csquotes}

% 符号字体
\usepackage{fontawesome5}

%%% Local Variables: 
%%% mode: latex
%%% TeX-master: "../main.tex"
%%% End:


% 进行必要的设置
% 载入字体(需要安装相应字体)
\setmainfont{LibertinusSerif}[% 英文字体
  Extension      = .otf,
  UprightFont    = *-Regular,
  BoldFont       = *-Bold,
  ItalicFont     = *-Italic,
  BoldItalicFont = *-BoldItalic,
  Scale          = 1.0]
\setmonofont{Iosevka}[Scale=1.0]% 等宽字体,主要用于代码排版
\setmathfont{LibertinusMath-Regular.otf}% Iosevka数学字体,需要unicode-math支持
\setCJKmainfont{Source Han Serif SC}[ % 中文衬线字体,思源宋体
  UprightFont     = * SemiBold,
  BoldFont        = * Heavy,
  ItalicFont      = * Light,
  BoldItalicFont  = * Medium,
  RawFeature      = +fwid]
\setCJKsansfont{Source Han Sans SC}[ % 中文无衬线字体,思源宋体
  UprightFont     = * Medium,
  BoldFont        = * Heavy,
  ItalicFont      = * Light,
  BoldItalicFont  = * Normal,
  RawFeature      = +fwid]  
\setCJKmonofont{Sarasa Mono SC}[% 中文等宽字体,Sarasa Mono SC
  UprightFont     = * Medium,
  BoldFont        = * Medium,
  ItalicFont      = * Extralight,
  BoldItalicFont  = * Light,
  RawFeature      = +fwid]   
% 也可以直接使用字体名称加载中文字体,思源字体,请将字体置于fonts目录下
% \setCJKmainfont[Extension=.otf,
%     Path=fonts/,
%     UprightFont=NotoSerifCJKsc-Regular,
%     BoldFont=NotoSerifCJKsc-Bold,
%     ItalicFont=NotoSerifCJKsc-Regular,
%     BoldItalicFont=NotoSerifCJKsc-Bold,
%     ItalicFeatures=FakeSlant,
%     BoldItalicFeatures=FakeSlant]{NotoSerifCJKsc}
% % 无衬线字体,思源字体,请将字体置于fonts目录下         
% \setCJKsansfont[
%     Extension=.otf,
%     Path=fonts/,
%     UprightFont=NotoSansCJKsc-Regular,
%     BoldFont=NotoSansCJKsc-Bold,
%     ItalicFont=NotoSansCJKsc-Regular,
%     BoldItalicFont=NotoSansCJKsc-Bold,
%     ItalicFeatures=FakeSlant,
%     BoldItalicFeatures=FakeSlant]{NotoSansSC}
% % 中文等宽字体,思源字体,请将字体置于fonts目录下  
% \setCJKmonofont[
%     Extension=.otf,
%     Path=fonts/,
%     UprightFont=NotoSansMonoCJKsc-Regular,
%     BoldFont=NotoSansMonoCJKsc-Bold,
%     ItalicFont=NotoSansMonoCJKsc-Regular,
%     BoldItalicFont=NotoSansMonoCJKsc-Bold,
%     ItalicFeatures=FakeSlant,
%     BoldItalicFeatures=FakeSlant]{NotoSansMonoSC}

%% 自定义相关的名称宏命令
%% ==================================================
%% \newcommand{\yourcommand}[参数个数]{内容}
% 西北农林科技大学各单位名称
\newcommand{\nwafu}{西北农林科技大学}
\newcommand{\cie}{信息工程学院}
\newcommand{\ca}{农学院}
\newcommand{\cpp}{植物保护学院}
\newcommand{\ch}{园艺学院}
\newcommand{\cast}{动物科技学院}
\newcommand{\cvm}{动物医学院}
\newcommand{\cf}{林学院}
\newcommand{\claa}{风景园林艺术学院}
\newcommand{\cnre}{资源环境学院}
\newcommand{\cwrae}{水利与建筑工程学院}
\newcommand{\cmee}{机械与电子工程学院}
\newcommand{\cfse}{食品科学与工程学院}
\newcommand{\ce}{葡萄酒学院}
\newcommand{\cls}{生命科学学院}
\newcommand{\cs}{理学院}
\newcommand{\ccp}{化学与药学院}
\newcommand{\cem}{经济管理学院}
\newcommand{\cm}{马克思主义学院}
\newcommand{\dfl}{外语系}
\newcommand{\iec}{创新实验学院}
\newcommand{\ci}{国际学院}
\newcommand{\dpe}{体育部}
\newcommand{\cvae}{成人教育}
\newcommand{\iswc}{水土保持研究所}

%% 签署春秋学期日期命令
\newcommand{\tomonth}{
  \the\year 年\the\month 月
}


\newcommand{\tomonthen}{
  \ifcase\the\month
  \or January%
  \or February%
  \or March%
  \or April%
  \or May%
  \or June%
  \or July%
  \or August%
  \or September%
  \or October%
  \or November%
  \or December%
  \fi, \the\year
}

\newcommand{\tosemester}{
  \the\year 年\ 
  \ifcase\the\month
  \or 秋%
  \or 春%
  \or 春%
  \or 春%
  \or 春%
  \or 春%
  \or 春%
  \or 夏%
  \or 秋%
  \or 秋%
  \or 秋%
  \or 秋%
  \fi 
}

\newcommand{\tosemesteren}{  
  \ifcase\the\month
  \or Autumn%
  \or Spring%
  \or Spring%
  \or Spring%
  \or Spring%
  \or Spring%
  \or Summer%
  \or Autumn%
  \or Autumn%
  \or Autumn%
  \or Autumn%
  \or Autumn%
  \fi, \the\year
}

% 插图路径设置
% ==================================================
\graphicspath{{figs/}}%图片所在的目录
% ==================================================

% 载入需要的TiKZ库
\usetikzlibrary{chains}

%% 设置绘制目录结构的宏及参数
\usepackage[edges]{forest}
\definecolor{folderbg}{RGB}{124,166,198}
\definecolor{folderborder}{RGB}{110,144,169}
\newlength\Size
\setlength\Size{4pt}
\tikzset{%
  folder/.pic={%
    \filldraw [draw=folderborder, top color=folderbg!50, bottom color=folderbg] (-1.05*\Size,0.2\Size+5pt) rectangle ++(.75*\Size,-0.2\Size-5pt);
    \filldraw [draw=folderborder, top color=folderbg!50, bottom color=folderbg] (-1.15*\Size,-\Size) rectangle (1.15*\Size,\Size);
  },
  file/.pic={%
    \filldraw [draw=folderborder, top color=folderbg!5, bottom color=folderbg!10] (-\Size,.4*\Size+5pt) coordinate (a) |- (\Size,-1.2*\Size) coordinate (b) -- ++(0,1.6*\Size) coordinate (c) -- ++(-5pt,5pt) coordinate (d) -- cycle (d) |- (c) ;
  },
}
\forestset{%
  declare autowrapped toks={pic me}{},
  declare boolean register={pic root},
  pic root=0,
  pic dir tree/.style={%
    for tree={%
      folder,
      %font=\ttfamily,
      grow'=0,
      s sep=1.0pt,
      font=\small \sffamily,
      %fit=band,
      %ysep = 1.0pt,
      inner ysep = 2.6pt,
    },
    before typesetting nodes={%
      for tree={%
        edge label+/.option={pic me},
      },
      if pic root={
        tikz+={
          \pic at ([xshift=\Size].west) {folder};
        },
        align={l}
      }{},
    },
  },
  pic me set/.code n args=2{%
    \forestset{%
      #1/.style={%
        inner xsep=2\Size,
        pic me={pic {#2}},
      }
    }
  },
  pic me set={directory}{folder},
  pic me set={file}{file},  
}
%% ==================================================

%%% Local Variables: 
%%% mode: latex
%%% TeX-master: "../main.tex"
%%% End: 


% Information
\title[凤岗主题]{\Large 西北农林科技大学 {\LaTeX}  Beamer 主题}

\subtitle{凤岗主题 V\ 2.0}

\author[N. Geng]{耿楠}

\date{\tosemester} % 也可以使用类似\date[2017/04/20]{\zhdate{2017/04/20}}的
              % 方式指定时间

\institute[智能媒体]{智能媒体实验室}

\titlegraphic{%
  \vspace{3.0cm}
  \qrcode[hyperlink, height=1.6cm]{https://github.com/registor/nwafuBeamerTheme}}

\begin{document}

\begin{frame}[plain]
  \maketitle
\end{frame}

\section{简介}

\begin{frame}{简介}{魔改Metropolis主题}
  \begin{itemize}
    \item 改了颜色(用了 \texttt{cncolours.sty})
    \item 加入 \texttt{pgfornament-han} 汉风纹样元素
  \end{itemize}
\end{frame}

\begin{frame}{简介}{为什么叫「凤岗」?}
  \begin{itemize}
    \item 以城市名字命名主题,是 Beamer 的一个传统
    \item 工作在杨凌,北塬上有个地方叫\alert{凤岗}
  \end{itemize}
\end{frame}

\begin{frame}[plain, standout]
作为强调的一个\enquote{standout} 页面
\end{frame}

\section{充版面}

\begin{frame}[allowframebreaks]{区块}{汉风区块结构}

  \begin{block}{Metropolis 走极简风}
    因此「凤岗」主题也走极简风。
  \end{block}

  \begin{exampleblock}{Metropolis 走极简风}
    因此「凤岗」主题也走极简风。
  \end{exampleblock}

  \begin{alertblock}{Metropolis 走极简风}
    因此「凤岗」主题也走极简风。
  \end{alertblock}

  \begin{theorem}[Metropolis 走极简风]
    因此「凤岗」主题也走极简风。
  \end{theorem}

  \begin{proof}[Metropolis 走极简风]
    因此「凤岗」主题也走极简风。
  \end{proof}
\end{frame}

%%%%%%%%%%%%%%%%%%%%%%%%%%%%%% 关于我们 %%%%%%%%%%%%%%%%%%%%%%%%%%%%%%%%%%%
\section[关于我们]{关于我们}
\subsection[联系方式]{联系方式}
\begin{frame}[fragile]{关于我们}{联系方式}
  % colour options
  \definecolor{seplinecolour}{HTML}{357f2d} % green
  \definecolor{iconcolour}{HTML}{2f3142} % dark
  \definecolor{textcolour}{HTML}{2f3142} % dark
  \definecolor{jobtitlecolour}{HTML}{474a65} % light dark

  % define some lengths for internal spacing
  \newlength{\seplinewidth} \setlength{\seplinewidth}{2cm}
  \newlength{\seplineheight} \setlength{\seplineheight}{1pt}
  \newlength{\seplinedistance} \setlength{\seplinedistance}{0.3cm}
  \begin{center}
      \begin{tikzpicture}[font=\small]
      % name
      \matrix[every node/.style={anchor=center,font=\huge},anchor=center] (name) {
        \node{耿\hspace{\ccwd}楠}; \\
        %\node{\color{jobtitlecolour}\normalsize\textit{教授}}; \\
      };
      % sep line 1
      \node[below=0.3\seplinedistance of name] (hl1) {};
      \draw[line width=\seplineheight,color=seplinecolour] (hl1)++(-\seplinewidth/2,0) -- ++(\seplinewidth,0);
      % contact info
      \matrix [below=\seplinedistance of hl1,%
               column 1/.style={anchor=center,color=iconcolour},%
               column 2/.style={anchor=west}] (contact) {
        \node{\faGlobe}; & \node{\url{http://cie.nwsuaf.edu.cn/szdw/}};\\
        \node{\faBuilding}; & \node{西北农林科技大学信息工程学院计算机科学系};\\ 
        \node{\faEnvelope}; & \node{nangeng@nwafu.edu.cn; nangeng@qq.com};\\
        %\node{\faQq}; & \node{970291228};\\
        %\node{\faPhone}; & \node{15829540966}; \\
        \node{\faGithub}; & \node{\url{https://github.com/registor/}}; \\
      };
      sep line 2
      \node[below=\seplinedistance of contact] (hl2) {};
      \draw[line width=\seplineheight,color=seplinecolour] (hl2)++(-\seplinewidth/2,0) -- ++(\seplinewidth,0);
      % interests
      \matrix [below=\seplinedistance of hl2,
         every node/.style={anchor=center,font=\Large}]
         (interests) {
        \node{\faCode}; & \node{\faCoffee}; &
        \node{\faLock}; & \node{\faWrench}; &
        \node{\faCameraRetro}; \\
      };
    \end{tikzpicture}
  \end{center}    
\end{frame}

\begin{frame}[plain, standout]
  感谢聆听!\\
  欢迎多提宝贵意见和建议
\end{frame}

\end{document}

%%% Local Variables:
%%% mode: latex
%%% TeX-master: t
%%% End:
