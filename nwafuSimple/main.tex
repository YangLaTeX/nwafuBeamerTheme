% \documentclass[10pt]{beamer}
\documentclass[xcolor=svgnames, t, aspectratio=169]{ctexbeamer}

\usetheme[
%%% options passed to the outer theme
%    progressstyle=corner,   % either fixedCircCnt, movCircCnt, or corner
%    rotationcw,                 % change the rotation direction from counter-clockwise to clockwise
%    shownavsym                  % show the navigation symbols
  ]{nwafusimple}

% 载入需要的宏包(在当前目录的settings目录下)
% 加载宏包
%===================注意======================%
% 在调用beamer.cls宏包后,以下宏包将自动调用,
% 不应单独调用这些宏包,以免发生冲突
% amsfonts, amsmath, amssymb, amsthm, 
% enumerate, geometry, graphics, graphicx, 
% hyperref, url, 
% ifpdf, keyval, xcolor, xxcolor
% =============================================%

% 参考文献宏包
% \usepackage[backend=biber,autolang=hyphen,style=gb7714-2015ay,
% doi=false,url=false,isbn=false]{biblatex}
\usepackage[backend=biber,autolang=hyphen,style=gb7714-2015,
gbtype=true,gbalign=gb7714-2015,
doi=false,url=false,isbn=false]{biblatex}

% Extra packages for the demo:
\usepackage{booktabs}
\usepackage{colortbl}
\usepackage{ragged2e}
\usepackage{schemabloc}

% 加载需要的宏包
\usepackage{csquotes}

% 符号字体
\usepackage{fontawesome5}

%%% Local Variables: 
%%% mode: latex
%%% TeX-master: "../main.tex"
%%% End:


% 进行必要的设置(在当前目录的settings目录下)
% 载入字体(需要安装相应字体)
\setmainfont{LibertinusSerif}[% 英文字体
  Extension      = .otf,
  UprightFont    = *-Regular,
  BoldFont       = *-Bold,
  ItalicFont     = *-Italic,
  BoldItalicFont = *-BoldItalic,
  Scale          = 1.0]
\setmonofont{Iosevka}[Scale=1.0]% 等宽字体,主要用于代码排版
\setmathfont{LibertinusMath-Regular.otf}% Iosevka数学字体,需要unicode-math支持
\setCJKmainfont{Source Han Serif SC}[ % 中文衬线字体,思源宋体
  UprightFont     = * SemiBold,
  BoldFont        = * Heavy,
  ItalicFont      = * Light,
  BoldItalicFont  = * Medium,
  RawFeature      = +fwid]
\setCJKsansfont{Source Han Sans SC}[ % 中文无衬线字体,思源宋体
  UprightFont     = * Medium,
  BoldFont        = * Heavy,
  ItalicFont      = * Light,
  BoldItalicFont  = * Normal,
  RawFeature      = +fwid]  
\setCJKmonofont{Sarasa Mono SC}[% 中文等宽字体,Sarasa Mono SC
  UprightFont     = * Medium,
  BoldFont        = * Medium,
  ItalicFont      = * Extralight,
  BoldItalicFont  = * Light,
  RawFeature      = +fwid]   
% 也可以直接使用字体名称加载中文字体,思源字体,请将字体置于fonts目录下
% \setCJKmainfont[Extension=.otf,
%     Path=fonts/,
%     UprightFont=NotoSerifCJKsc-Regular,
%     BoldFont=NotoSerifCJKsc-Bold,
%     ItalicFont=NotoSerifCJKsc-Regular,
%     BoldItalicFont=NotoSerifCJKsc-Bold,
%     ItalicFeatures=FakeSlant,
%     BoldItalicFeatures=FakeSlant]{NotoSerifCJKsc}
% % 无衬线字体,思源字体,请将字体置于fonts目录下         
% \setCJKsansfont[
%     Extension=.otf,
%     Path=fonts/,
%     UprightFont=NotoSansCJKsc-Regular,
%     BoldFont=NotoSansCJKsc-Bold,
%     ItalicFont=NotoSansCJKsc-Regular,
%     BoldItalicFont=NotoSansCJKsc-Bold,
%     ItalicFeatures=FakeSlant,
%     BoldItalicFeatures=FakeSlant]{NotoSansSC}
% % 中文等宽字体,思源字体,请将字体置于fonts目录下  
% \setCJKmonofont[
%     Extension=.otf,
%     Path=fonts/,
%     UprightFont=NotoSansMonoCJKsc-Regular,
%     BoldFont=NotoSansMonoCJKsc-Bold,
%     ItalicFont=NotoSansMonoCJKsc-Regular,
%     BoldItalicFont=NotoSansMonoCJKsc-Bold,
%     ItalicFeatures=FakeSlant,
%     BoldItalicFeatures=FakeSlant]{NotoSansMonoSC}

%% 自定义相关的名称宏命令
%% ==================================================
%% \newcommand{\yourcommand}[参数个数]{内容}
% 西北农林科技大学各单位名称
\newcommand{\nwafu}{西北农林科技大学}
\newcommand{\cie}{信息工程学院}
\newcommand{\ca}{农学院}
\newcommand{\cpp}{植物保护学院}
\newcommand{\ch}{园艺学院}
\newcommand{\cast}{动物科技学院}
\newcommand{\cvm}{动物医学院}
\newcommand{\cf}{林学院}
\newcommand{\claa}{风景园林艺术学院}
\newcommand{\cnre}{资源环境学院}
\newcommand{\cwrae}{水利与建筑工程学院}
\newcommand{\cmee}{机械与电子工程学院}
\newcommand{\cfse}{食品科学与工程学院}
\newcommand{\ce}{葡萄酒学院}
\newcommand{\cls}{生命科学学院}
\newcommand{\cs}{理学院}
\newcommand{\ccp}{化学与药学院}
\newcommand{\cem}{经济管理学院}
\newcommand{\cm}{马克思主义学院}
\newcommand{\dfl}{外语系}
\newcommand{\iec}{创新实验学院}
\newcommand{\ci}{国际学院}
\newcommand{\dpe}{体育部}
\newcommand{\cvae}{成人教育}
\newcommand{\iswc}{水土保持研究所}

%% 签署春秋学期日期命令
\newcommand{\tomonth}{
  \the\year 年\the\month 月
}


\newcommand{\tomonthen}{
  \ifcase\the\month
  \or January%
  \or February%
  \or March%
  \or April%
  \or May%
  \or June%
  \or July%
  \or August%
  \or September%
  \or October%
  \or November%
  \or December%
  \fi, \the\year
}

\newcommand{\tosemester}{
  \the\year 年\ 
  \ifcase\the\month
  \or 秋%
  \or 春%
  \or 春%
  \or 春%
  \or 春%
  \or 春%
  \or 春%
  \or 夏%
  \or 秋%
  \or 秋%
  \or 秋%
  \or 秋%
  \fi 
}

\newcommand{\tosemesteren}{  
  \ifcase\the\month
  \or Autumn%
  \or Spring%
  \or Spring%
  \or Spring%
  \or Spring%
  \or Spring%
  \or Summer%
  \or Autumn%
  \or Autumn%
  \or Autumn%
  \or Autumn%
  \or Autumn%
  \fi, \the\year
}

% 插图路径设置
% ==================================================
\graphicspath{{figs/}}%图片所在的目录
% ==================================================

% 载入需要的TiKZ库
\usetikzlibrary{chains}

%% 设置绘制目录结构的宏及参数
\usepackage[edges]{forest}
\definecolor{folderbg}{RGB}{124,166,198}
\definecolor{folderborder}{RGB}{110,144,169}
\newlength\Size
\setlength\Size{4pt}
\tikzset{%
  folder/.pic={%
    \filldraw [draw=folderborder, top color=folderbg!50, bottom color=folderbg] (-1.05*\Size,0.2\Size+5pt) rectangle ++(.75*\Size,-0.2\Size-5pt);
    \filldraw [draw=folderborder, top color=folderbg!50, bottom color=folderbg] (-1.15*\Size,-\Size) rectangle (1.15*\Size,\Size);
  },
  file/.pic={%
    \filldraw [draw=folderborder, top color=folderbg!5, bottom color=folderbg!10] (-\Size,.4*\Size+5pt) coordinate (a) |- (\Size,-1.2*\Size) coordinate (b) -- ++(0,1.6*\Size) coordinate (c) -- ++(-5pt,5pt) coordinate (d) -- cycle (d) |- (c) ;
  },
}
\forestset{%
  declare autowrapped toks={pic me}{},
  declare boolean register={pic root},
  pic root=0,
  pic dir tree/.style={%
    for tree={%
      folder,
      %font=\ttfamily,
      grow'=0,
      s sep=1.0pt,
      font=\small \sffamily,
      %fit=band,
      %ysep = 1.0pt,
      inner ysep = 2.6pt,
    },
    before typesetting nodes={%
      for tree={%
        edge label+/.option={pic me},
      },
      if pic root={
        tikz+={
          \pic at ([xshift=\Size].west) {folder};
        },
        align={l}
      }{},
    },
  },
  pic me set/.code n args=2{%
    \forestset{%
      #1/.style={%
        inner xsep=2\Size,
        pic me={pic {#2}},
      }
    }
  },
  pic me set={directory}{folder},
  pic me set={file}{file},  
}
%% ==================================================

%%% Local Variables: 
%%% mode: latex
%%% TeX-master: "../main.tex"
%%% End: 
  
  
\title[Beamer 主题] % (可选,仅当标题过长时使用)
{西北农林科技大学 {\LaTeX} Beamer NWAFUsidebar 主题}

\subtitle{v\ 2.0} % 也可以是一个其它的名字
\date{\tosemester} % 也可以使用类似\date[2017/04/20]{\zhdate{2017/04/20}}的
              % 方式指定时间

\author[N. Geng] % (可选,仅当有多个作者时使用)
{耿楠}

\institute[智能媒体实验室] % 可选项,在每页边栏的底部显示
{% 显示在标题页
  \cie
  
  % 在此要有一个空行,否则会在大学和国家之间产生额外的空白(I do not
  % 不知道为什么;( )
}
% ==================================================

% 编译控制==================================================设定只输出
% 选定标签的章节,加快编译速度
% \includeonlylecture{lec:introduction}

% 设定仅编译的帧,加快编译速度
% \includeonlyframes{testframe}
% ==================================================

\begin{document}
% 封面
{\nwafuwavesbg%
\begin{frame}[plain,noframenumbering] % plain选项删除页眉页脚
  \titlepage
\end{frame}
}
%%%%%%%%%%%%%%%%

% Beamer 内容
\section{简介}
\begin{frame}{简介}{内容简介}
  \begin{itemize}
  \item 西北农林科技大学 {\LaTeX} \alert{Beamer 主题}
    \begin{itemize}
    \item 简单易用
    \item 标准、规范
      \begin{itemize}
      \item 高质量
      \item 高效率
      \end{itemize}
    \item \alert{所想即所得}
    \end{itemize}
  \end{itemize}
\end{frame}
%%%%%%%%%%%%%%%%

\subsection{协议}
% 协议
\begin{frame}{简介}{协议}
  \begin{itemize}
  \item 所有logo的版权属于西北农林科技大
    学\href{http://www.nwafu.edu.cn}{http://www.nwafu.edu.cn}
  \item 若署名为西北农林科技大学,则可以使用这些logo
  \item 请遵守 GNU 通用公共协议 v.3 (GPLv3)
    \begin{itemize}
    \item 详
      见:
      \href{http://www.gnu.org/licenses/}{http://www.gnu.org/licenses/}
    \item 可以发布和修改本主题中的任何内容
    \end{itemize}
  \end{itemize}
\end{frame}
%%%%%%%%%%%%%%%%

\section{安装}
% 通用安装
\begin{frame}{安装}{概述}
  \begin{itemize}
  \item 文件构成
    \begin{enumerate}
    \item {\tt beamerthemenwafusimple.sty}
    \item {\tt beamerinnerthemenwafusimple.sty}
    \item {\tt beamerouterthemenwafusimple.sty}
    \item {\tt beamercolorthemenwafusimple.sty}
    \item {\tt beamerfontthemenwafusimple.sty}
    \end{enumerate}
  \item 全局安装
    \begin{itemize}
    \item 拷贝到本地\LaTeX 目录树中
    \end{itemize}
  \item 本地安装
    \begin{itemize}
    \item 将5个主题文件拷贝到当前工作文件夹
    \end{itemize}    
  \end{itemize}
\end{frame}

\subsection{GNU/Linux}
% GNU/Linux中的安装
\begin{frame}{安装}{GNU/Linux}
  \begin{block}{Ubuntu中的TeX Live}
    \begin{enumerate}
    \item 将 {\tt <dirstruct>} 拷贝到本地{\LaTeX}目录树的根目录. 默认是\\
      {\tt \textasciitilde /texmf}\\
      如果根目录不存在,则创建该目录。 符号 {\tt \textasciitilde} 表示
      家目录, 例如:{\tt /home/<username>}
    \item 在终端中运行如下命令\\
      {\tt \$ texhash \textasciitilde /texmf}
    \end{enumerate}
  \end{block}
\end{frame}
%%%%%%%%%%%%%%%%

% Microsoft Windows
\begin{frame}{安装}{Microsoft Windows}
  \begin{block}{Windows中的TeX Live}
    假设使用默认目录(在高级 TeX Live 安装中,可以更改latex目录树的根目
    录)。
    \begin{enumerate}
    \item 将 {\tt <dirstruct>} 拷贝到本地{\LaTeX}目录树的根目录\\
      {\tt \%USERPROFILE\%\textbackslash texmf}\\
      如果不存在,则创建. XP的默认目录是 {\tt \%USERPROFILE\%} 是\\
      {\tt c:\textbackslash Document and
        Settings\textbackslash<username>},\\
      Vista及更高版本是\\
      {\tt c:\textbackslash Users\textbackslash<username>}
    \item 打开 TeX Live 管理器对话框选择 'Actions'中的'Update filename
      database',并执行.
    \end{enumerate}
  \end{block}
\end{frame}
%%%%%%%%%%%%%%%%

\subsection{Mac OS X}
% Mac OS X的安装
\begin{frame}{安装}{Mac OS X}
  \begin{block}{Mac OS X中的 MacTeX}
    将 {\tt <dirstruct>} 拷贝到本地latex目录树的根目录. 默认是\\
    {\tt \textasciitilde /Library/texmf}\\
    如果不存在,则创建. 符号 {\tt \textasciitilde} 表示家目录, 例
    如:{\tt /home/<username>}
  \end{block}
\end{frame}
%%%%%%%%%%%%%%%%

\subsection{宏包依赖}
% 宏包依赖
\begin{frame}{安装}{宏包依赖}
  除需要 Beamer 类外,本主题需要调用两个宏包
  \begin{itemize}
  \item TikZ\footnote{TikZ 是一个绘制图形的杰出宏包.请参
      考\href{http://www.texample.net/tikz/examples/}{在线示
        例} 或
      \href{http://tug.ctan.org/tex-archive/graphics/pgf/base/doc/generic/pgf/pgfmanual.pdf}{pgf
        用户手册}. }
  \item calc
  \end{itemize}
  这些宏包是{\LaTeX}的通用宏包。
\end{frame}
%%%%%%%%%%%%%%%%

\section{用户接口}
\subsection{主题及选项}
% 主题和选项列表
\begin{frame}{用户接口}{加载主题和主题选项}
  \begin{block}{演示文稿主题}
    加载主题只需要输入\\
    {\tt \textbackslash usetheme[<选项>]\{nwafusimple\}}\\
    与加载其它主题方法一致,本主题会加载内部、外部和颜色主题并且可以传
    递 {\tt <选项>} 参数.
  \end{block}
  \begin{block}{内部主题}
    使用如下命令加载内部主题\\
    {\tt \textbackslash useinnertheme\{nwafusimple\}}\\
    内部主题无参数.
  \end{block}
\end{frame}
%%%%%%%%%%%%%%%%

\begin{frame}{用户接口}{加载主题和主题选项}
  \begin{block}{外部主题}
    使用如下命令加载主题\\
    {\tt \textbackslash usetheme[<选项>]\{nwafusimple\}}\\
    目前,主题的参数有:
    \begin{itemize}
      \scriptsize
    \item {\tt progressstyle=}: 进度条样式
      \begin{itemize}
      \item {\tt fixedCircCnt}: 固定圆形进度条,只显示当前帧计数值
      \item {\tt movCircCnt}: 动态圆形进度条,当前帧计数值沿圆环移动
      \item {\tt corner}: 右下角页脚以xx/xx形式显示进度
      \end{itemize}
    \item {\tt rotationcw}: 圆形进度条由逆时针变换为顺时针方向
    \item {\tt shownavsym}: 是否显示导航符号
    \end{itemize}
  \end{block}
\end{frame}
%%%%%%%%%%%%%%%%
\subsection{编译}
% 编译
\begin{frame}{用户界面}{编译}
  \begin{block}{演示文稿的编译}
    本主题需要至少编译 \alert{3} 次,以保证正确处理页码计数器的数字。
  \end{block}
\end{frame}
%%%%%%%%%%%%%%%%

\subsection{主题修改}
% 如何修改主题
{\setbeamercolor{nwafusimple}{fg=gray!50,bg=gray}
  \setbeamercolor{simple}{bg=red!20}
  \setbeamercolor{structure}{fg=red}
  \setbeamercolor{frametitle}{use=structure,fg=structure.fg,bg=red!5}
  \setbeamercolor{normal text}{bg=gray!20}
  \begin{frame}{用户界面}{主题修改}
    \begin{itemize}
    \item 主题设置了默认的字体、颜色和布局。
    \item 请参考beamer用户手册。
    \item 例如,在这一页中,使用如下方式修改了主题元素
      \begin{itemize}
      \item 修改边栏颜色:\\
        {\tt \textbackslash
          setbeamercolor\{nwafusidebar\}\{fg=gray!50,bg=gray\}} {\tt
          \textbackslash setbeamercolor\{sidebar\}\{bg=red!20\}}
      \item 修改结构元素颜色:\\
        {\tt \textbackslash setbeamercolor\{structure\}\{fg=red\}}\\
      \item 修改帧标题文本颜色和背景颜色: {\tt \textbackslash
          setbeamercolor\{frametitle\}\{use=structure,
          fg=structure.fg,bg=red!5\}}
      \item 修改文本背景颜色{\tt \textbackslash setbeamercolor\{normal
          text\}\{bg=gray!20\}}
      \end{itemize}
    \end{itemize}
  \end{frame}}
%%%%%%%%%%%%%%%%

\subsection{波浪背景}
% nwafu波浪背景图案
\begin{frame}{用户界面}{波浪背景图案}
  \begin{block}{波浪背景图案}
    \begin{itemize}
    \item 可以在任意一个单独帧中用下述方法添加背景图案\\
      {\tt \{\textbackslash nwafuwavesbg\\
        \textbackslash begin\{frame\}[<选项>]\{帧标题\}\{帧子标题\}\\
        \ldots\\
        \textbackslash end\{frame\}\}}
    \end{itemize}
  \end{block}
\end{frame}
%%%%%%%%%%%%%%%%

\subsection{宽屏支持}
% 宽屏支持
\begin{frame}{用户界面}{宽屏支持}
  \begin{block}{宽屏支持}
    新式投影仪,甚至是现代的电视机已支持如 16:10 或 16:9 模式的宽屏模式.
    Beamer (>= v. 3.10) 支持多种演示文稿的缩放比例。根据Beamer 用户手册
    (v. 3.10)的77页的8.3节的说明,可以使用如下选项设置显示比例\\
    {\tt\textbackslash documentclass[aspectratio=1610]\{beamer\}}\\
    这一命令设置为 16:10. 也可以是 169, 149, 54, 43 (默认).
  \end{block}
\end{frame}
%%%%%%%%%%%%%%%%

%%%%%%%%%%%%%%%%

\section{主题应用}
{\setbeamercolor{nwafusidebar}{fg=gray!50,bg=gray}
 \setbeamercolor{sidebar}{bg=red!20}
 \setbeamercolor{structure}{fg=red}
 \setbeamercolor{frametitle}{use=structure,fg=structure.fg,bg=red!5}
 \setbeamercolor{normal text}{bg=gray!20}
 \subsection{文件夹结构}
\begin{frame}{主题应用}{文件夹结构}
  \begin{itemize}
  \item 本地安装时的工作文件构成
  \end{itemize}
  \begin{center}
    \scalebox{0.65}{
      \begin{forest}
        pic dir tree,
        pic root,
        for tree={% folder icons by default; override using file for file icons
          directory,
        },
        [jobname【工作根目录】%, 
          [nwafulogo【学校校徽图标,\alert{必须存在},且置于根目录】
            [nwafu-circle.pdf【学校圆形透明Logo】, file
            ]
            [h\_bar.pdf【学校横条形中英文透明Logo】, file
            ]
            [nwafu\_waves.pdf【波纹背景透明图】, file
            ]
            [nwafu\_logo\_cie.png【信息工程学院透明Logo】, file
            ]
          ]
          [settings【自定义命令、环境、引入宏包等\LaTeX 源文件,可根据需要调整】        
            [format.tex【自定义命令、环境、参数设置等】, file
            ]
            [packages.tex【引入宏包】, file
            ]
          ]
          [beamerthemenwafusimple.sty【主题主文件,\alert{必须存在},且置于根目录】, file
          ]
          [beamerinnerthemenwafusimple.sty【内部主题文件,\alert{必须存在},且置于根目录】, file
          ]
          [beamerouterthemenwafusimple.sty【外部主题文件,\alert{必须存在},且置于根目录】, file
          ]
          [beamercolorthemenwafusimple.sty【颜色主题文件,\alert{必须存在},且置于根目录】, file
          ]
          [beamerfontthemenwafusimple.sty【字体主题文件,\alert{必须存在},且置于根目录】, file
          ]
          [main.tex【主控文件,\emph{必须存在},且置于根目录】, file
          ]
          [Makefile【make命令需要的脚本文件,若不执行make命令,可以不需要】, file
          ]
          [.latexmkrc【latexmk命令需要的脚本文件,若不执行latexmk命令,可以不需要】, file
          ]
        ]
      \end{forest}
    }
  \end{center}
\end{frame}}
%%%%%%%%%%%%%%%%

\section{问题反馈}
\subsection{错误, 意见和建议}
% 问题、意见和建议
\begin{frame}{问题反馈}{错误、意见和建议}
  \begin{itemize}
  \item 本主题中会存在错误,如果发现了错误,请联系我进行改进
    \begin{itemize}
    \item \alert{再小的问题也是问题}
    \end{itemize}
  \item 如果你有好建议和使用体验改善意见,请联系我进行改进
  \end{itemize}
\end{frame}
%%%%%%%%%%%%%%%%

%%%%%%%%%%%%%%%%%%%%%%%%%%%%%% 关于我们 %%%%%%%%%%%%%%%%%%%%%%%%%%%%%%%%%%%
\section[关于我们]{关于我们}
\subsection[联系方式]{联系方式}
\begin{frame}[fragile]{关于我们}{联系方式}
  % colour options
  \definecolor{seplinecolour}{HTML}{357f2d} % green
  \definecolor{iconcolour}{HTML}{2f3142} % dark
  \definecolor{textcolour}{HTML}{2f3142} % dark
  \definecolor{jobtitlecolour}{HTML}{474a65} % light dark

  % define some lengths for internal spacing
  \newlength{\seplinewidth} \setlength{\seplinewidth}{2cm}
  \newlength{\seplineheight} \setlength{\seplineheight}{1pt}
  \newlength{\seplinedistance} \setlength{\seplinedistance}{0.3cm}
  \begin{center}
      \begin{tikzpicture}[font=\small]
      % name
      \matrix[every node/.style={anchor=center,font=\huge},anchor=center] (name) {
        \node{耿\hspace{\ccwd}楠}; \\
        %\node{\color{jobtitlecolour}\normalsize\textit{教授}}; \\
      };
      % sep line 1
      \node[below=0.3\seplinedistance of name] (hl1) {};
      \draw[line width=\seplineheight,color=seplinecolour] (hl1)++(-\seplinewidth/2,0) -- ++(\seplinewidth,0);
      % contact info
      \matrix [below=\seplinedistance of hl1,%
               column 1/.style={anchor=center,color=iconcolour},%
               column 2/.style={anchor=west}] (contact) {
        \node{\faGlobe}; & \node{\url{http://cie.nwsuaf.edu.cn/szdw/}};\\
        \node{\faBuilding}; & \node{西北农林科技大学信息工程学院计算机科学系};\\ 
        \node{\faEnvelope}; & \node{nangeng@nwafu.edu.cn; nangeng@qq.com};\\
        %\node{\faQq}; & \node{970291228};\\
        %\node{\faPhone}; & \node{15829540966}; \\
        \node{\faGithub}; & \node{\url{https://github.com/registor/}}; \\
      };
      sep line 2
      \node[below=\seplinedistance of contact] (hl2) {};
      \draw[line width=\seplineheight,color=seplinecolour] (hl2)++(-\seplinewidth/2,0) -- ++(\seplinewidth,0);
      % interests
      \matrix [below=\seplinedistance of hl2,
         every node/.style={anchor=center,font=\Large}]
         (interests) {
        \node{\faCode}; & \node{\faCoffee}; &
        \node{\faLock}; & \node{\faWrench}; &
        \node{\faCameraRetro}; \\
      };
    \end{tikzpicture}
  \end{center}    
\end{frame}

% 封底
{\nwafuwavesbg
\begin{frame}[plain,noframenumbering]
  \finalpage{谢谢你使用该{\LaTeX} Beamer主题!\\欢迎多提宝贵意见和建议!}
\end{frame}
}
%%%%%%%%%%%%%%%%

\end{document}

%%% Local Variables:
%%% mode: latex
%%% TeX-master: t
%%% End:
